\documentclass[UTF8]{article}
\usepackage{graphicx}
\usepackage{indentfirst}
\usepackage{subfigure}
\usepackage[UTF8]{ctex}
\usepackage{amsmath}
\usepackage{enumitem}
\usepackage{geometry}
\usepackage{float}
\usepackage{wrapfig}

\geometry{b5paper,left=1.5cm, right=1.5cm, top=1.7cm, bottom=1.7cm}

\title{learning}
\author{John Smith}
\date{\today}

\begin{document}
\section{A}

\subsection{Ag,Au}

Ag,Au单质溶于 $ \rm K CN$ 溶液
\begin{equation}
\begin{aligned}
    \rm \nonumber 
    &4Ag+8 {CN}^{-} +2H_{2}O+O_{2} = 4[Ag(CN)_{2}]^{-} +4OH^{-} \\
    &4Au+8 {CN}^{-} +2H_{2}O+O_{2} = 4[Au(CN)_{2}]^{-} +4OH^{-}
\end{aligned}
\end{equation}

Ag在空气中变黑
\begin{equation}
    \begin{aligned}
        \rm \nonumber
        2Ag^{+}+S^{2-} = Ag_{2}S 
    \end{aligned}
\end{equation}

\textbf{Ag和Au的提取} 
\begin{itemize}[leftmargin=50pt]
    \setlength{\itemsep}{0pt}
    \setlength{\parsep}{0pt}
    \setlength{\parskip}{0pt}
    \item $ \rm NaCN $浸取
    \item 还原
    \begin{itemize}
        \item[] Zn还原:$\rm 2Au[(CN)_{2}]^{-}+Zn=[Zn(CN)_{4}]^{2-}+2Au$
        \item[] 电解还原:
        \begin{itemize}
            \item[] 阴极$\rm [Au(CN)_{2}]^{-} +Zn = [Zn(CN)_{4}]^{2-}+2Au$ 
            \item[] 阳极$\rm CN^{-}+2OH^{-} -2e^{-} = CNO^{-}+H_{2}O$
        \end{itemize}
    \end{itemize}
\end{itemize}

$\rm Ag_{2}O$可由反应$\rm Ag^{+}+OH^{-} \rightarrow AgOH \rightarrow Ag_{2}O + H_{2}O$得到

$\rm AgOH$可在强碱与可溶性$\rm Ag^{+}$盐的乙醇溶液在低温得到
\[
\rm Ag_{2}O+4NH_{3}+H_{2}O = 2Ag[(NH_{3})_{2}]^{+}+2OH^{-}
\]

$\rm 2AgBr \overset { \text{光照} } { = } 2Ag +Br_{2}$可用于洗胶卷

\[
\rm 3Mg(OH)_{2} +2[AuCl_{4}]^{-} = 2Au(OH)_{3}+3Mg^{2+}+8Cl^{-}
\]

\subsection{As}

$\rm AsH_{3}$剧毒,其中As显-3价。

$\rm As_{2}O_{3}$两性偏酸,俗称砒霜

\textbf{马氏试砷法:}将锌单质、盐酸、试样混合,把生成的气体通入热玻璃管,若玻璃管壁上有黑亮的砷镜,则试样中有$\rm AsH_{3}$

原理:
\[
\rm As_{2}O_{3} +6Zn +12HCl(aq)=2AsH_{3}+6ZnCl_{3}+3H_{2}O
\]
\[
\rm 2AsH_{3} \overset{\bigtriangleup}{ = } 2As+3H_{2}
\]

将锌、盐酸、试样混合,把生成的气体通入热玻璃管,若玻璃管壁上有黑亮的砷镜,则试样中有$\rm As_{2}O_{3}$

$\rm AsH_{3}$在空气中自燃

\[
\rm 2As_{2}H_{3}+12AgNO_{3}+3H_{2}O
\]

古氏试砷法:
\[
\rm 2AsH_{3}+12AgNO_{3}+3H_{2}O=As_{2}O_{3}+12HNO_{3}+12Ag \downarrow
\]

现象:溶液中有黑色沉淀生成

\subsection{B}
硼单质有晶体和无定形两类

硼单质可以在空气中燃烧,大量放热

硼砂($\rm Na_{2}B_{4}O_{7}$)溶于水,再用$\rm H_{2}SO_{4}$调pH可得$H_{3}BO_{3}$

\[ \rm Na_{2}B_{4}O_{7} + H_{2}SO_{4} +5H_{2}O = 4H_{3}BO_{3} +Na_{2}SO_{4} \]

乙硼烷的制取:

\[ \rm 3LiAlH_{4} +4BCl_{3} \overset{乙醚中}{=} 2B_{2}H_{6} +3LiCl +3AlCl_{3}\]

含硼元素化合物的特殊结构:

\textbf{略}

无机苯中含有$ \Pi _{6}^{6}$. 乙硼烷中有

\[
\rm 4B+3O_{2} \overset{\text{燃烧}}{=} 2B_{2}O_{3}
\]

\end{document}
